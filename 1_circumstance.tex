
%\mode<presentation>{

%------- Beamer theme ------

\usetheme[progressbar=none]{metropolis}
\metroset{sectionpage=none}

\makeatletter
\setlength{\metropolis@titleseparator@linewidth}{1pt}
\setlength{\metropolis@progressonsectionpage@linewidth}{1pt}
\setlength{\metropolis@progressinheadfoot@linewidth}{1pt}
\makeatother

%\setbeamercovered{transparent}
%\usefonttheme[onlymath]{serif}
\usefonttheme{serif}
\usefonttheme{professionalfonts}

%\setbeamertemplate{headline}{}
%\setbeamertemplate{footline}{}
\setbeamertemplate{blocks}[rounded][shadow=true]
%\setbeamertemplate{navigation symbols}{}
\setbeamertemplate{itemize items}[circle]  % ball, circle
%\setbeamertemplate{enumerate items}[circle]
\setbeamertemplate{enumerate items}[default]
%\setbeamertemplate{section in toc}[circle]
\setbeamertemplate{section in toc}
{\leavevmode\leftskip=2ex%
  \llap{%
    \usebeamerfont*{section number projected}%
    \usebeamercolor{section number projected}%
    \begin{pgfpicture}{-1ex}{0ex}{1ex}{2ex}
      \color{darkorange}
      \pgfpathcircle{\pgfpoint{0pt}{.65ex}}{1.25ex}
      \pgfusepath{fill}
      \pgftext[base]{\color{fg}\inserttocsectionnumber}
    \end{pgfpicture}\kern1.25ex%
  }%
  \inserttocsection\par}

\setbeamersize{text margin left=0.5cm, text margin right=0.5cm}


\definecolor{darkblue}{rgb}{0.02,0.24,0.44}
\definecolor{darkorange}{rgb}{0.80,0.4,0.07}
\definecolor{lightgray}{rgb}{0.78,0.78,0.78}
\definecolor{darkgray}{rgb}{0.30,0.30,0.30}
\definecolor{darkgreen}{rgb}{0.02,0.18,0.20}
\definecolor{footer}{rgb}{0.5,0.2,0.5}

\setbeamercolor{frametitle}{fg=darkblue,bg=black!2}
\setbeamercolor{section in head/foot}{bg=black,fg=darkorange}
\setbeamercolor{lower separation line head}{bg=darkblue}
\setbeamertemplate{section in head/foot shaded}{\color{lightgray}
\usebeamertemplate{section in head/foot}}

\setbeamercolor{title in head/foot}{bg=darkgreen,fg=black!5}
\setbeamercolor{section in toc}{fg=darkorange}

%\setbeamertemplate{section in toc}[sections numbered]
%\setbeamertemplate{subsection in toc}[subsections numbered]
%\setbeamertemplate{section in toc}[ball unnumbered]
%\setbeamertemplate{subsection in toc}[ball unnumbered]

% change the style of headline
\setbeamertemplate{headline}{
\begin{beamercolorbox}[ht=4.25ex]{section in head/foot} %,dp=0.0ex
   \usebeamerfont{title in head/foot}
   \vskip2pt\insertsectionnavigationhorizontal{\textwidth}{}{}\vskip2pt
   %\vskip2pt\insertnavigation{\paperwidth}\vskip2pt
\end{beamercolorbox}
\begin{beamercolorbox}[colsep=1.5pt]{lower separation line head} %
\end{beamercolorbox}
}


\setbeamertemplate{frametitle}{%
    \nointerlineskip%
    \begin{beamercolorbox}[wd=\paperwidth,ht=3ex,dp=1.25ex]{frametitle}
        \hspace*{1.5ex}\insertframetitle %
    \end{beamercolorbox}%
    \vspace*{-1ex}
}

%\setbeamertemplate{footline}{%
%  \hfill%
%  %\usebeamercolor[fg]{page number in head/foot}%
%  \usebeamercolor[footer]{page number in head/foot}%
%  \usebeamerfont{page number in head/foot}%
%  \insertframenumber \,/\,\inserttotalframenumber
%  \kern1em\vskip3pt%
%}

\setbeamertemplate{footline}
{%
  \leavevmode%
  \begin{beamercolorbox}[wd=0.25\paperwidth,ht=2.25ex,dp=1ex,leftskip=.3cm,rightskip=.3cm plus1fil]{title in head/foot}%
    \usebeamerfont{title in head/foot} ~~\insertshortauthor~(\insertshortinstitute) \hfill
  \end{beamercolorbox}%
  \begin{beamercolorbox}[wd=0.5\paperwidth,ht=2.25ex,dp=1ex,leftskip=.3cm,rightskip=.3cm plus1fil]{title in head/foot}%
    \usebeamerfont{title in head/foot}\centering \insertshorttitle
  \end{beamercolorbox}%
  \begin{beamercolorbox}[wd=0.25\paperwidth,ht=2.25ex,dp=1ex,leftskip=.3cm,rightskip=.3cm plus1fil]{title in head/foot}%
    \usebeamerfont{title in head/foot} \hfill
    \insertframenumber\,/\,\inserttotalframenumber~~%
  \end{beamercolorbox}%
  \vskip0pt%
}



%--------- 宏包  ----------
\usepackage[UTF8,noindent]{ctex}
%\usepackage[english]{babel}
\usepackage{amsmath,amssymb,version}
\usepackage{graphicx,fancybox,mathrsfs,multirow}
\usepackage{booktabs}
\usepackage{epsfig,epstopdf}
\usepackage{url,hyperref}
\usepackage{tabularx,array,makecell}
\usepackage{color,xcolor}
\usepackage{cases}
\usepackage{tcolorbox}
\usepackage{mathtools}
\usepackage{tikz}




%---------- 定义行距 ----------
%\renewcommand{\baselinestretch}{1.15}

%---------- 定义表格新命令 ----------
\newcolumntype{P}[1]{>{\centering \arraybackslash}p{#1}}
\newcolumntype{L}{>{\quad}X}
\newcolumntype{C}{>{\centering \arraybackslash}X}
\newcolumntype{R}{>{\raggedright \arraybackslash}X}

%---------- 设置字体  ----------
\setbeamerfont{normal text}{family=\rmfamily}
\setbeamerfont{frametitle}{family=\rmfamily}
\setbeamerfont{title}{family=\sffamily}
\setbeamerfont{subtitle}{family=\sffamily}
\setbeamerfont{institute}{family=\sffamily}
\setbeamerfont{author}{family=\sffamily}
\setbeamerfont{date}{family=\sffamily}
\setbeamerfont{headline}{family=\sffamily} %,series=\bfseries
\setbeamerfont{footline}{family=\sffamily} %\kaishu
\setbeamerfont{section in toc}{family=\rmfamily}
\setbeamerfont{subsection in toc}{family=\rmfamily}
\AtBeginDocument{\usebeamerfont{normal text}}

%\setbeamerfont{section in head/foot}{family=\sffamily}%,series=\bfseries
\setbeamercolor{title}{fg=white}
\setbeamercolor{subtitle}{fg=black!10}
\setbeamercolor{institute}{fg=white}
\setbeamercolor{author}{fg=white}
\setbeamercolor{date}{fg=white}

\setbeamertemplate{caption}[numbered]
\numberwithin{figure}{section}
\numberwithin{table}{section}
\numberwithin{equation}{section}

%---------- 定理设置  --------------
\setbeamertemplate{theorems}[numbered]
%\newtheorem{theorem}{Theorem}
\newtheorem{theorem}{定理}
\numberwithin{theorem}{section}
%\newtheorem{definition}{Definition}
\newtheorem{definition}{定义}
\numberwithin{definition}{section}
%\newtheorem{lemma}{Lemma}
\newtheorem{lemma}{引理}
\numberwithin{lemma}{section}
%\newtheorem{proposition}{Proposition}
\newtheorem{proposition}{命题}
\numberwithin{proposition}{section}
%\newtheorem{corollary}{Corollary}
\newtheorem{corollary}{推论}
\numberwithin{corollary}{section}
\theoremstyle{example}
%\newtheorem{example}{Example}
\newtheorem{example}{例}
%\numberwithin{example}{section}
\renewenvironment{proof}[1][证明]{\textbf{#1}:~}{\qed\par}


%---------- 调节公式的间距 ----------
%\AtBeginDocument{
%	\setlength{\abovedisplayskip}{4pt plus 1pt minus 1pt}
%	\setlength{\belowdisplayskip}{4pt plus 1pt minus 1pt}
%	\setlength{\abovedisplayshortskip}{2pt}
%	\setlength{\belowdisplayshortskip}{2pt}
%	\setlength{\arraycolsep}{2pt}
%}


\makeatletter
\newcommand\HUGE{\@setfontsize\Huge{28}{32}}
\makeatother


%\AtBeginSection[]{
%\begin{frame}
%  \frametitle{Outline} %\small
%  \vskip -5pt
%  \hspace*{1.5em}
%  \parbox[t]{.95\textwidth}{
%  \begin{minipage}[c][0.6\textheight]{\textwidth}
%  \tableofcontents[currentsection,currentsubsection,subsectionstyle=show/show/shaded]
%  \end{minipage} }
%  \addtocounter{framenumber}{-1}
%\end{frame}
%}


% 设置图片添加位置
\graphicspath{{./figures/}}